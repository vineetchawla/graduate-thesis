\chapter{Flight Data}
The very first step for the insurance platform is gathering data. For accurate prediction, historical data for each flight landing in Germany will be required. There are no open source datasets available for German flights currently on internet. For this reason the dataset had to created from scratch. There are numerous websites that keep historical data of flights. There were two steps to be followed to get this data. Firstly, we would create a list of every flight that lands in Germany. Once we have that list, we try to get the historical data of each flight.

\section{List of all flights landing in Germany}
Getting the list of every flight landing in Germany was not a burdensome task. The first step was getting the list of airports in Germany. With a quick search on Flightradar24, small list of 56 airports that are used for commercial and charter flights in Germany was available\footnote{\url{https://www.flightradar24.com/data/airports/germany}}. The airports that just handled chartered flights were ignored for this dataset. For each airport of Germany, a list was available that had records of each flight landing in that airport. Each record for each airport was copied into an excel file, with the most important flight details like destination, origin, airline and flight ID. On completing the same process for each airport a record of about 3000 flights was created that came into Germany from more than 200 cities in the world.

\section{Plausible Sources for historical flight data}
The next step was to get the historical data of at least 3 months for each flight in our newly created list. As this was going to be a huge dataset that could not be created/curated manually, the requirements for this task were defined as follows:
\begin{enumerate}
    \item The flight data source should have at least 3 months of historical data of flights, especially the actual departure, actual arrival and flight delay.
    \item The flight data source should preferably have a developer friendly API to automate the whole process of getting the data
    \item The API calls or subscription costs for the flight data source should be reasonable
\end{enumerate}

Based on the above mentioned criteria, the following websites that kept flight records were evaluated.

\subsection{Flightstats}
Flightstats states it provides the historical data of flights to it's registered users. They also have a developer API but that too is limited to use by registered users. Registring is not a big issue but in case of FlightStats, it is not easy to register. The user has to give specific domain where the data collected will be used, assure FlightStats there will be no commercial use and give details of what kind of data would be required. On making the request multiple times, my request was rejected. Soon a call was made by Flightstats sales team that offered me to provide the complete dataset I required, historical records of six months of each German flight, as an excel file. The happiness of receiving the complete and clean data was shortlived though as the company asked 2000€ for the dataset, discounted price for students, which is otherwise double of what was quoted. Hence Flightstats was not considered anymore.

\subsection{Flightaware}
Flightaware on the other hand had a easy to register and use developer API which provides interface for REST requests. The developer key is provided free of cost. The free account of flightaware.com provides historical records of 14 days only. At \$20 per month, the flight history provided is 5 months\footnote{\url{https://flightaware.com/commercial/premium/}}.But on further research, it was noted that  the historical data was just for display. Their API, FlightXML 3 doesn't provide any particular support for making REST requests for historical data of a flight\footnote{\url{[https://flightaware.com/commercial/flightxml/pricing_class.rvt}}.
Hence Flightaware too was not selected

\subsection{Flightradar24}
Flightradar24 is one of the most famous sites on internet with mobile apps available for iPhone and Android. Flightradar24 does have historical data of 6 months for their Gold tier customers but no API to get that data. Due to being out of options, this site was selected on the basis of cost/performance basis.

\section{Getting data}
Not having a API for getting flight history was an obstacle in collecting data but it could be overcome by scraping the flight data instead. As we could see last 6 months of flight data on the webpage.
Even though many websites specifically consider web scraping illegal, flightradar24 only rejects web-scraping of data if that data is to be used commercially. Hence this website being a college project, doesn't fit under the commercial category.

\subsection{scraping}
Web scraping is basically looking into the source code of a webpage and saving the data you require[EC]. As we already had the flight ID of all the flights in Germany, the list for all webpages which would have to scraped was created using the URL template \url{https://www.flightradar24.com/data/flights/<flight_id>/}. 
Once we had all the web-URLs to be scraped, the actual scraping was started. \\As the whole website was to be created in Python so  the scraping was also done in Python.Python has a very handy utility that provides a very easy to use library for web scraping called Scrapy. 
In the most basic sense, we get the whole webpage, and then save the desired elements of the URL in JSON format. The important variables for us were 'From', 'To', flight time, STD, AD, SA, AA. All of this data was saved in a csv file.
 

\section{Data Variables}
All the data we received from scraping was saved in a csv file.
The total records we saved were about 385,000 but many of the observations were missing important variables. In this section we go through each variable and then later on dwell into the cleaning of data so that it's suitable for analysis.
\begin{enumerate}
    \item {Departure Airport}
    \\Origin Airport of the flight. This value can be of any airport in the world that has a direct flight to a German airport
    \item {Arrival Airport}
    \\Destination airport of the flight. It will always be one of the German airports
    \item {Standard Arrival}
    \\The fixed arrival time of a flight time based on the flight schedule.
    \item {Actual Arrival}
    \\The actual arrival of the flight, that might be earlier or most probably at or after the standard arrival of the flight
    \item {Standard Departure}
    \\The fixed departure time of a flight time based on the flight schedule.
    \item {Actual Departure}
    \\The actual departure of the flight, that might be earlier or most probably at or after the standard departure of the flight
    \item {Airline}
    \\The flight company
    \item {Flight ID}
    \\The The flight ID is a unique number provided by International Air Transport Organisation to each flight.
    \item {Aircraft}
    \\The type of aircraft for the flight
\end{enumerate}

\section{Data Cleanup and missing values substitution}
On opening the csv file with all the stored data, it could be noted that a lot of data had some missing values. A lot of data we had was  repetitive data, for ex arrival and destination airports stay same for the flight no matter the date of flight. But some of the data like flight time would change everyday. Hence different strategies were created for cleaning missing values for each variable.
Excel was selected for data cleanup due to its ease of use and data visualisation capabilities

\subsection{Departure Airport}
Departure Airport stays constant for each Flight ID. Many of the flight IDs had different airports on different dates, and some were missing the value. This was rectified for each of the 3000 different flights.

\subsection{Arrival Airport}
The same strategy was applied to Arrival airport for cleanup as we used for Departure Airport. But this variable had couple of more problems that needed different fix:
\begin{enumerate}
    \item Flight diverted
    \\Arrival airport too remains constant for each Flight ID except in case the flight is diverted. This was a special case and only happened nine times in our dataset. All these entries were removed. \item Connecting Flights
    \\There are many indirect flight that connect to German airports. For sake of simplicity, it was decided to keep only flight lading in Germany. For ex, if a flight flew from Delhi to Moscow, and then from Moscow to Frankfurt, only flight that came from Moscow was considered and rest were discarded. 
\end{enumerate} 

\subsection{Standard Arrival}
The Standard Arrival of a flight does not necessarily stay the same all year. For ex., it changes on the basis of daylight savings or even as a business decision by an airport or Airline. Hence instead of changing all values of standard departure same for complete history of a flight, the values were copied from the upper or lower cell in the data belonging to the same flight. 

\subsection{Standard Departure}
The same strategy as for Standard Arrival was used for cleaning up Standard Departure too.

\subsection{Actual Arrival}
Arguably the most important variable that we have in our dataset. The actual flight delay will be calculated using this variable. Any row that doesn't has this value is removed from the dataset.

\subsection{Actual Departure}
Any values that were missing from actual departure, were assumed to be the same as standard departure as the variable was of not much importance to us.
\subsection{Airline}
The airline was corrected on the basis of the upper or lower cells in the dataset. 

\subsection{Flight ID}
Even though an important variable, it's not at all required for our data analysis. The variable was left as is.

\subsection{Aircraft}
For cleaning the missing data in Aircraft, it was assumed that a flight uses same aircraft unless a pattern is noticed where each subsequent flight uses a different aircraft.

\subsection{Flight duration}
Another very important variable in our dataset. And missing value was calculated by subtracting Actual departure from Actual Arrival. All the values were also converted into seconds instead of keeping them in Time format.

\section{Adding new variables}
After cleaning the dataset, we decided to add new variables to the dataset which would hopefully make the predictions better aligned to our end goal

\subsection{Flight delay bins}
Out insurance rates will vary based on our flight delay prediction. To ease this process and give customer a better perspective about the insurance payout, we create 4 different categories of flight Instead of using the exact time of delays. 
\begin{enumerate}
    \item 0 if flight is on time or less than 15 minutes late
    \item 1 if flight is 15 or minutes but less than an hour late
    \item 2 if flight is more than 15 minutes but less than an hour late
    \item 3 if flight is an hour or more late
\end{enumerate}

\subsection{Weekend}
A simple variable. Using Weekday formula in excel, we create a binary variable which is true if the flight is on weekends.

\subsection{Time Blocks}
Instead of using exact arrival time of our flight, we create multiple time blocks of 2 hours so that this variable can be used as factor for our classification algorithms. A flight at 8:30 in the morning was classified in block 8 and flight at 15:30 is classified in block 16.

\section{Reducing Levels}
One problem that we face in classification algorithms when analysing such a big dataset is number of categories of each variable. We have more than 50 airlines and 80 aircraft in our dataset. Algorithms in R like random forest don't accept more than 32 categories.[EC] Boosting algorithms like XGBoost on the other hand require indicator columns instead of categorical variable[EC]. Hence each column with n categories is converted to n columns. This makes the dataset too large. Reducing levels will help in this scenario too.

\subsection{Airline}
We had 67 different airlines in our dataset with Lufthansa being the most popular with 124940 flight records. We can see in the figure  that we have many airlines with insignificant number of flights. 
%insert image
Hence we combine the airlines with lowest number of flights in a new airline category called other\_airline

\begin{figure}[ht]
    \centering
    \includegraphics[width=\textwidth]{Figures/Aircraft_orig_levels.png}
    \caption{Reduced categories of Airlines with number of airline}
    \label{fig:airline2}
\end{figure}

\subsection{Aircraft}
We had about 80 different kind of aircraft in our dataset, with Airbus A320 being the most common choice, with over 81000 flights. We can see in the figure \ref{fig:aircraft1} that we have many aircraft with insignificant number of flights.

\begin{figure}[ht]
    \centering
    \includegraphics[width=\textwidth]{Figures/Aircraft_orig_levels.png}
    \caption{Original categories of Aircraft with number of flights}
    \label{fig:aircraft1}
\end{figure}

We combine all the aircraft with the lowest number of variables in a new value called other\_aircraft, as seen in figure \ref{fig:aircraft2}.

\begin{figure}[ht]
    \centering
    \includegraphics[width=\textwidth]{Figures/Aircraft_reduced_levels.png}
    \caption{Reduced categories of Aircraft with number of flights}
    \label{fig:aircraft2}
\end{figure}

\subsection{Departure Airport}
The departure airport has 287 different departure airports from all over the world. Many of the airports have non-insignificant number of flights so converting a majority of flights to a single category would have been counter-productive. Hence we left the variable as is to measure if this variable is any important to our prediction or could be left out.

\section{End result - dataset}

After cleaning the data, substituting the missing values, reducing the number of levels and adding new variables to help in prediction, the summary of the dataset is as seen in figure \ref{fig:summary_flights}:

\begin{figure}[ht]
    \centering
    \includegraphics[width=\textwidth]{Figures/summary_flight_data.png}
    \caption{Final summary of our flight dataset}
    \label{fig:summary_flights}
\end{figure}