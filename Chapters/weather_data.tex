\chapter{Weather Data}
To further make the training dataset more comprehensive, it was decided to integrate the weather data in with the historical flight details. The plan was to add a multitude of the weather characteristics of the Arrival airport at the time of the arrival of the flight.

\section{Plausible sources for weather data}
Thankfully, collecting historical weather data is much more widespread than collecting historical flight data. As a result, there were many options to choose from, all of which provided a rich developer API.
\\The considerations for choosing weather data provider were based on these two conditions
\begin{enumerate}
    \item Limitations on making API calls
    \item Cost of making API calls
\end{enumerate}
The following three providers were shortlisted and evaluated on the above stated criteria:

\subsection{Apixu}
Although not a famous website, apixu provides historical data of upto a year ago. The subscription of a premium plan of 30 Euros per month is required for the privilege though. It's a reasonable cost but the API has limitations of 300,000 calls per month. For normal use cases this is enough, but for this dataset, it was not enough.

\subsection{openweathermap}
Openweathermap is another famous weather data provider. They have multiple plans for providing historical weather data, but the one that would be suitable for this project would be 950 Euros per month\footnote{\url{http://openweathermap.org/price}}. This is way too expensive for the data so this option was taken out of consideration too.

\subsection{DarkSky API}
Darksky is another weather data provider, lending its data to a famous weather site called \url{Forecast.io}. There is no monthly subscription for this API. Instead the costs are based on the API calls made by the developer. They provide a possibility of 1000 calls every day for free with every subsequent call charged at just \EUR{.00001}. This totalled to only 35 euros for complete data that was required, hence this was a suitable choice. 

\section{Getting data}
Using API from Darksky is easy but the parameters it required were not available in the correct format, or missing altogether from the dataset. 
\begin{enumerate}
    \item Latitude and Longitude
    \\ The latitude and longitude for each airport in Germany was not available anywhere so this information was collected manually using Google maps. In Google maps, once we point to a location, the latitude and longitude of that point are available in its URL. This process was repeated for all the airports in the dataset and then synced to the dataset for each flight.
    \item Time
    \\Even though the time of flight and the date for each record was available, this API required time in UNIX epoch format, number of seconds that have elapsed since 1\textsuperscript{st} January 1970. Using date and actual arrival time of each flight, a new variable was created which had the corresponding epoch format time.
\end{enumerate}

After these new variables were integrated into the dataset, the REST API calls were generated. As with other API calls, the script to get the data was written in python and the data was integrated right into the CSV file containing flight dataset. The following data variables were saved from the requests for each flight.

\begin{enumerate}
    \item Summary
    \\A human-readable text summary of the weather at the time.
    \item Humidity
    \\The relative humidity, between 0 and 1, inclusive
    \item Cloud cover
    \\The percentage of sky occluded by clouds, between 0 and 1, inclusive
    \item Visibility
    \\The average visibility in kilometres, capped at 10 kilometres
    \item Wind Speed
    \\The wind speed in kilometres per hour
    \item Wind Bearing
    \\The direction that the wind is coming from in degrees, with true north at 0° and progressing clockwise. (If windSpeed was zero, then this value was not defined.)
    \item Apparent Temperature
    \\The apparent (or “feels like”) temperature in degrees Celsius.
    \item Dew Point
    \\The percentage of sky occluded by clouds, between 0 and 1, inclusive
    \item Pressure
    \\The sea-level air pressure in millibars.
    \item Precipitation Intensity
    \\The intensity (in inches of liquid water per hour) of precipitation occurring at the given time. This value is conditional on probability (that is, assuming any precipitation occurs at all) for minute data points, and unconditional otherwise.
\end{enumerate}

\section{Missing values cleanup}
As the data was already clean before requesting, the data for each of the flight record was received. The python script used to request the temperature also made sure each value was in sync with the actual flight records.
But there were still many values that were missing in the dataset. The missing variables were less than 100 for all the weather variables. Instead of substituting mean values, to fix this, it was assumed that the weather at same time and location on a day would be very similar to the day before or after. Hence this strategy was used and the missing values were substituted with values above or below the data cell. Rows with outliers were removed from the dataset and a row with multiple 0 values in 3 or more weather variables was removed too.