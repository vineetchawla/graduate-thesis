\chapter{Blockchain}

Blockchain is the fundamental technology of this thesis. Blockchain technology was first described in a paper written by Satoshi Nakamoto in his highly influential paper introducing Bitcoin. Before delving into how blockchain forms part of our product, we will go through exactly how blockchain was first implemented in Bitcoin, the cryptocurrency that started the blockchain and arguably a financial revolution.

\section{Introduction to Bitcoin}
Blockchain technology was first described in a paper written by Satoshi Nakamoto in his highly influential paper introducing Bitcoin \cite{Nakamoto2008Bitcoin:System}.In the paper, he described it as a purely peer-to-peer version of electronic cash that would allow online payments to be sent directly from one party to another without going through a financial institution. Up until then, these ideas were theoretical and had not been successfully implemented \cite{Chaum1983BlindPayments}. The Bitcoin currency had three main components that we discuss below.\cite{Economist2013HowWork}

\subsection{Blockchain}
The block chain is a shared public ledger on which the entire Bitcoin network relies. All confirmed transactions are included in the block chain. This way, Bitcoin wallets can calculate their spendable balance and new transactions can be verified to be spending bitcoins that are actually owned by the spender. The integrity and the chronological order of the block chain are enforced with cryptography.

\subsection{Private Keys}
A transaction is a transfer of value between Bitcoin wallets that gets included in the block chain. Bitcoin wallets keep a secret piece of data called a private key or seed, which is used to sign transactions, providing a mathematical proof that they have come from the owner of the wallet. The signature also prevents the transaction from being altered by anybody once it has been issued. All transactions are broadcast between users and usually begin to be confirmed by the network in the following 10 minutes, through a process called mining.

\subsection{Mining}
Mining is a distributed consensus system that is used to confirm waiting transactions by including them in the block chain. It enforces a chronological order in the block chain, protects the neutrality of the network, and allows different computers to agree on the state of the system. To be confirmed, transactions must be packed in a block that fits very strict cryptographic rules that will be verified by the network. These rules prevent previous blocks from being modified because doing so would invalidate all following blocks. Mining also creates the equivalent of a competitive lottery that prevents any individual from easily adding new blocks consecutively in the block chain. This way, no individuals can control what is included in the block chain or replace parts of the block chain to roll back their own spends.

For this project, we will focus on the Blockchain, it's inner workings and how it helps us in achieving our goal of automating flight insurance.

\section{Blockchain}
As discussed above, Blockchain is what is now known as public distributed ledger, designed in the first place to solve the double spending problem, that is, to establish consensus in a decentralised network over who owns what and what has already been spent\cite{Nakamoto2008Bitcoin:System}. 

A distributed ledger is essentially an asset database that can be shared across a network of multiple sites, geographies or institutions. All participants within a network can have their own identical copy of the ledger. Any changes to the ledger are reflected in all copies in minutes, or in some cases, seconds. The assets can be financial, legal, physical or electronic. The security and accuracy of the assets stored in the ledger are maintained cryptographically through the use of ‘keys’ and signatures to control who can do what within the shared ledger. Entries can also be updated by one, some or all of the participants, according to rules agreed by the network.\cite{Walport2015DistributedChain}
So how does bLockchain help in keeping records of Bitcoin transactions.  Blockchain enables Bitcoin transactions to be aggregated in ‘blocks’ and these are added to a ‘chain’ of existing blocks using a cryptographic signature. The Bitcoin ledger is constructed in a distributed and ‘permission-less’ fashion, so that anyone can add a block of transactions if they can solve a new cryptographic puzzle to add each new block. The incentive for solving the puzzle, or mining in Bitcoin terms, is that each miner gets a reward for each time the block they mined is added to the blockchain.\cite{Nakamoto2008Bitcoin:System}

\section{Benefits of Blockchain}
What we have when abstracting a blockchain network to a certain level is a distributed, self-authenticating, time-stamped store of data\cite{MonaxBlockchains}. Indeed, the core design of a blockchain node is an elegant way in which to overcome many challenges in distributed systems.
\subsection{Resilient Data Management System}
Blockchain clients allow for the development of distributed systems which do not rely on what traditional databases call ‘master-slave’ clusters. This drastically increases the resiliency of blockchain networks as a data management solution.
In a blockchain network, there is not even a notion of master-slave relationships between the nodes in the cluster. Instead, blockchain networks utilise the idea of peer nodes and consensus models to resolve the current world state of the data.
This allows for a fluid membership to the “truth creating” consortium of computers which in turn increases fault tolerance and resiliency. Breaking the data-driven transactions into blocks allows the consensus of the database to be negotiated in a reasonable manner rather than on a per-transaction basis.
\subsection{Increased Verifiability}
In addition, blockchain networks allow for transactional certainty. Traditional databases store the current world state of the data, and if they are programmed to do so, have additional entries covering previous transactions within the data store. In addition, traditional databases are also able to maintain logs of the history of the interactions.
Blockchain networks are designed differently in that the logs of the transactions with the data set are used to formulate the world state of the data. The use of cryptographic authentication of time-stamped blocks of transactions allows the entire network the benefit of certainty of the entire transactional history.
The general blockchain design not only requires that the transactional history of the data store is captured, but that it is cryptographically certain once there is sufficient consensus within the network.

\section{Potential of Blockchain}
Blockchain technology has attracted attention as the basis of cryptocurrencies such as Bitcoin, but its capabilities extend far beyond that, enabling existing technology applications to be vastly improved and new applications never previously practical to be deployed. Also known as distributed ledger technology, blockchain is expected to revolutionise industry and commerce and drive economic change on a global scale because it is immutable, transparent, and redefines trust, enabling secure, fast, trustworthy, and transparent solutions that can be public or private. It could empower people in developing countries with recognised identity, asset ownership, and financial inclusion; and it could avert a repeat of the 2008 financial crisis, support effective health care programs, improve supply chains and, perhaps, clean up unethical behaviour in high-value businesses such as diamond trading.\cite{Underwood2016BlockchainBitcoin}

\subsection{Impact on Financial Industry}
The area where many see Blockchain first changing the current financial services is the back-office handling of transactions. When a financial institution sells a syndicated loan or derivative, the recording of the transaction is time consuming and involves burdensome back-office processes. These processes rely on negotiated contracts with the numerous associated lawyers and contact between the parties to finish the transaction. On average, it can take 20 days to settle a syndicated loan trade. These back-end activities are also costly to the financial institutions. This is in contrast to the front-end systems at financial institutions, where millions are spent to achieve a nanosecond of competitive advantage. In addition, financial institutions, due to regulatory requirements, are dealing with greater requirements for reporting, transparency, and dissemination of data. They need a technological breakthrough to help solve these problems. Blockchain can be the breakthrough that can streamline these financial transactions. There are estimates that Blockchain could save financial institutions at least \$20 billion annually in settlement, regulatory, and cross-border payment costs.\cite{Fanning2016BlockchainServices}

%%give citation to all below
\begin{itemize}
    \item Fintech startup R3, backed by over 40 global banks, is developing a standardised architecture for private ledgers that could significantly cut the cost and time of settling transactions.
    \item The Linux Foundation’s Hyperledger project is an industry initiative including tech giant IBM that is evolving open source technology and building the foundation of a standardised, production grade digital ledger
    \item Deloitte is working with clients and startups to develop solutions including Smart Identity, which can support banks’ regulatory client onboarding and Know Your Customer (KYC) processes, while individual financial institutions, insurance companies, exchanges, and solutions vendors also have thrown their weight behind blockchain
    \item Nasdaq is using its Linq blockchain technology to complete and record private securities transactions, and the Depository Trust \& Clearing Corporation, working with market participants and technology firm Axoni, is managing post-trade events for credit default swaps. Regulators are also interested in the technology, as its transparency and integrity allow market activity to be monitored in real time. 
\end{itemize}


\subsection{Impact on Commerce and Record Keeping}
\begin{itemize}
    \item Factom’s focus is on securing data. The company is participating in the Honduran land registry project and working on a number of projects in China, including data infrastructure for 80 smart cities, financial technology solutions, and integrating blockchain technology with electronic data notarization services to enhance integrity in information management. 
    \item Everledger’s focus is on the identity and legitimacy of objects. Blockchain works well here because its history cannot be changed and it enables trust by consensus. The company’s initial work provides a distributed ledger of diamond ownership and transaction history verification for owners, insurance companies, claimants, and law enforcement agencies. The system assists with prevention of fraud in the supply chain, but also helps consumers decide whether to buy particular diamonds.
\end{itemize}

\section{smart contract}
A smart contract is an automatable and enforceable agreement. Automatable by computer, although some parts may require human input and control. Enforceable either by legal enforcement of rights and obligations or via tamper-proof execution of computer code.\cite{Clack2016SmartDirections}
In simple words, it simply states a requirement that the contract must be enforceable without specifying what is the aspect being enforced; for smart legal contracts these might be complex rights and obligations, whereas for smart contract code what is being enforced may simply be the actions of the code.
These scripts are compiled into low level operation codes and stored in the blockchain’s data store at a particular address – which is determined when the contracts are deployed to the blockchain. When a transaction is sent to that address the distributed virtual machine on every full node of the blockchain network executes the script’s operation codes using the data which is sent with the transaction. \cite{MonaxContracts}

Smart contracts are modular, repeatable, autonomous scripts, which can be used to build applications for yourself, for a community, for a client, for a bounty, or even just for fun. They can be mixed and matched, and easy to iterate, rather like lego bricks combined with pre-set templates.
Smart contracts can be coded to reflect any kind of business or engineering logic which is data-driven: from actions as simple as up-voting a post on a forum, to the more complex such as loan collateralisation and futures contracts, to the highly complex such as repayment prioritisation on a structured note.

\section{Benefits of Smart Contracts}
As blockchain is a secure technology, so smart contracts can be more secure than traditional contract law. Also, they can reduce a number of transaction costs associated with contracting, since the blockchain cuts out any middlemen. However, the fact remains that the quality of the output depends on the quality of the input. Smart contracts are by no means magical constructs that understand user intent and are always flawless. If there is an oversight in the text, the result might be even more dramatic than in a traditional contract, because the rules of the smart contract are recorded in computer code and cannot be freely interpreted according to ‘the intent of the contract’, but only according to literal meaning.

\section{Original vision for the project}
As the project is outlined, the legal contract of insurance was initially supposed to be created as a smart contract. Right now there are multiple startups that are providing smart contract platforms. For the purpose of this project, two platforms were evaluated and tested for creating the legal agreement. We discuss these platforms and will go through the reasons that went into taking a decision to not use them. 

\subsection{Monax}
Monax is a platform that provides developers a free platform to build and run Smart contracts. The platform is based on Ethereum Virtual machines. There are no limitations on what the platform can do. The smart contracts can be for claims Management or Supply chain management etc. Monax also provide premium SDKs for sectors such as insurance but as they were expensive, they were not evaluated for the project. Instead the contract was developed from scratch using their platform with Solidity and Ethereum virtual machine. 
\subsubsection{Architecture}
For creating a permissioned blockchain on Monax platform, one machine is  required to act as Administrator with full rights and capabilities, and at least three machines are required to act as validator nodes, which just validate transactions and nothing else. The users that will buy insurance from us, will have a participant account, that has the rights to interact with the blockchain and nothing else.
\\ As Monax is a relatively new company, by the time this evaluation was going on, their product was in an alpha stage. Due to which there were many problems in setup. But as with small companies, it was easy to get in touch with the company and get all my issues pertaining to initial setup fixed. 
\\After setting up with one administrator node, 3 validator nodes and 2 participant nodes, we go through setting up certificates for each machine. Then we create account types and initial tokens before we mine for genesis block. 
\\ we give the command to create chains, which will create initial directories for each account we created
\\Once all the directories are created, we start the mining and our genesis block is created and mining begins.
\\The consensus of transactions between validators is done by using tendermint protocol, whose explanation is out of scope of this project.
\\Once our smart contract platform is running and mining transactions, we can create a smart contract in solidity app and commit transactions between participants.

But this is where the problems for excepting the project began:
\begin{itemize}
    \item The participants can't be created on the fly. My initial idea was whenever we create a user, that user should be classified as a participant.
    \item The complete blockchain runs in an Ethereum Virtual machine. There is no way to interact with the actual blockchain. Monax had a node.JS library to interact with the blockchain with no official support for Python. As the project is in Python, using a different and new language was not considered due to time constraints
\end{itemize}

\subsection{Corda}
Corda is arguably the most famous platform for smart Contracts. It is developed by R3 consortium, which is supported buy over 50 of the biggest financial institutions of the world.
\\A Corda network is an authenticated peer-to-peer network of nodes, where each node is a JVM run-time environment hosting Corda services and executing applications known as CorDapps.\cite{Hearn2016Corda:Ledger}. This whole network of nodes is in a permissioned network and each node uses point to point communication instead of global broadcast of requests. Instead of every node seeing every transaction between themselves, a node in Corda network can only see a transaction if the node was part of that transaction, but the node still can validate the transaction.
As each node is running a JVM run-time environment, the smart contract, or CorDapp has to written in a JVM compatible language, like Java or Kotlin.
\\ As in the case with Monax, it is not possible to interact with Corda Blockchain directly. But Corda has native support for Oracle. Oracles are network services that, upon request, provide commands that encapsulate a specific fact (e.g. the exchange rate at time x) and list the oracle as a required signer.\cite{OraclesDocumentation}. So I planned to create an oracle for getting flight status but realised that Oracle can't directly contact with REST API. This had to be done with a third party website such as oraclize.it. They provided tools necessary to get the data we want as a oracle service. 
\\The other reason for dropping the Corda platform was that it's architecture was in such a way that suited a limited number of financial institutions or entities which had multiple transactions in between them. This was contrary to the project idea for this thesis and hence the idea was dropped.

\section{Anchoring data to Blockchain}
Due to time constraints, the whole idea of creating smart contract on a blockchain was dropped and instead the plan for utilising the already existing Bitcoin blockchain was made. This plan was fulfilled by the concept called anchoring of data on Bitcoin blockchain. Anchoring in blockchain simply means adding your data to Bitcoin blockchain to provide irrefutable time-stamps and non-repudiation. 
We have multiple services that provide anchoring service for your data. For this project, Chainpoint was selected as they have a very easy to use (and still free) REST API called Tierion.
\\Chainpoint links a hash of your data to a blockchain and returns a timestamp proof. A Chainpoint service receives hashes which are aggregated together using a Merkle tree. The root of this tree is anchored in the Bitcoin and Ethereum blockchains. Throughout this process a Chainpoint proof is created and continually upgraded. The final Chainpoint proof defines a path of operations that cryptographically links your data to one or more blockchains.\cite{ChainpointStandard}
\\A Chainpoint proof is a JSON-LD document, that contains the information to cryptographically verify a piece of data is anchored to a blockchain. It proves the data existed at time it was anchored. Chainpoint proofs can be verified without reliance on a trusted third party.
\footnote{JSON-LD is a lightweight Linked Data format described here \url{https://json-ld.org/}}
\\Once the data is sent to Tierion, they anchor it to Bitcoin blockchain. But as mining of bitcoin blockchain takes time, the status and the blockchain receipt isn't provided until at least 10 minutes after the data is sent. Hence we will have to provide the user a manual control to update their status of insurance and receive the receipt.

\subsection{Benefits}
\begin{itemize}
    \item Proof
    \item Integrity of data
    \item Audit trail
    \item Immutable
    \item Privacy
\end{itemize}

\subsection{Integration with project}
The integration of data with Tierion was very as explained in the following sequence:
\begin{itemize}
    \item Once the user selects and agrees to our insurance rates, we create a JSON string with all the important data, especially the following variables:
    \item Once we have the data, we make a request to Tierion with the data. Tierion creates a hash of the data sent, and anchors it to the Bitcoin blockchain.
    \item We receive back an id of the data and the hash calculates.
    \item Initially the status of the blockchain in unpublished and blockchain receipt is null. However, within 10 minutes, the status is updated and the blockchain receipt is received.
\end{itemize}