\chapter{Website}

The website would be the main interface for any to communicate with. The complexity of the systems and data analysis should all be in the backend and the user should only see the simplest user interface making the experience as easy as possible. In this section, we will go through each of the section of the webpage and discuss the design decision and processes taking place.

\section{Technologies Used}
\subsection{Python}
\subsection{Flask}
\subsection{MySQL}
\subsection{IDE}
\subsection{Github}

\section{Website Structure}
\subsection{User Administration}
\subsection{SQL model}
The SQL model defines the actual database model that will be used by the website. This model contains multiple classes that corresponds to unique table in the MySQL. The first model contains the table for the important details for the website including username email password flight ID insurance ID blockchain receipt and other useful details. the in complete information about each flight of Germany. this is just used for the autocomplete function described in the following sections
\subsection{views}
\subsection{Jquery}
\subsection{REST API calls}
\subsection{Email Notifications}
\subsection{Blockchain}

\section{Website Details}
\subsection{Register Page}
\subsection{Login Page}
\subsection{Initial Dashboard}
\subsection{Flight Details}
\subsection{Insurance Rates}
\subsection{Dashboard}
\subsection{Blockchain Receipt}

\section{Cloud Provider}

\section{Security Features}