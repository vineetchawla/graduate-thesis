\chapter{Introduction}

Currently the insurance industry is beset with redundant and poorly optimised business processes. As such, insurance industry is not ranked high among consumers when it comes to reputation\cite{lauraMazzucaToops2014HarrisPropertyCasualty360}. One can think of many reasons due to which the industry hasn't considered upgrading significantly, although it comes down to multiple orthodoxies prevalent in insurance industry. Many insurers have treated these orthodoxies as entry barriers, effectively insulating them from being disrupted in a significant way by internal or external upstarts. They include the beliefs that\cite{Shaw2016InsurersDisrupted}:
\begin{itemize}
    \item Consumer familiarity with established insurers precludes wide-scale disruption by newcomers to the business.
    \item Insurance is often a complex, opaque, and even misunderstood product, which gives the industry's seasoned agent and broker sales force a considerable edge over would-be alternative distribution challengers.
    \item Insurers have effectively cornered the market on the data, models, and analytical talent to underwrite and price exposures as well as facilitate risk management.
    \item Since the premise of risk pooling is fundamental to the business of insurance, the massive capital reserves assembled by insurers cannot be easily replicated by new players.
\end{itemize}

But recent changes in the technology sector have woken up many companies to a possibility of disruption. Even though started as a base for bitcoin, blockchain has gained immense popularity as the next big thing with various implications to the traditional insurance industry.
\\ In this paper, the possibility of such a disruption will be evaluated. And the best way to evaluate will be to actually build such a service. The aim of this Master Thesis will be to create a product that fully exploits the plethora of innovations the technology world has seen in last decade and using them to create an insurance service that provides an efficiency and transparency standard for users that isn't possible with current business practices. The end-goal of this thesis will be to create a full fledged insurance service that provides complete transparency to the buyers about the insurance premium, payouts and claims, and automates the claim reimbursement process. This automation will be achieved via the combination of Machine learning and Blockchain, integrated tightly into a web based service.

\section{Existing Work}
Since realising Blockchain's potential, there have been many services that started providing commerce services based on Blockchain, but there is no particular website that provides a full fledged flight delay insurance capability, except for the following two:
\begin{enumerate}
    \item A similar project was created on Ethereum smart contract called Etherisc\footnote{\url{https://fdd.etherisc.com/} Requires an Ethereum client for functioning}. But using their insurance service needs an Ethereum client running on the user's machine and transactions are limited to Ether, severely limiting the functionality and reach of the platform.
    \item On 13\textsuperscript{th} September 2017, Insurance giant AXA launched a new website called fizzy\footnote{\url{fizzy.axa} Blockchain based Flight delay insurance by AXA} that provides the exact functionality as this prototype intends to provide via this project. Fizzy calculates the insurance rates based on the booked flight details and displays them to the user before the purchase of insurance. The claim process is also automatically triggered once the flight status is available with the help of smart contracts on Ethereum blockchain. Still it differs from this website in two fundamental ways. Firstly, the claim is possible only if the flight is delayed for more than 2 hours. And secondly, it is only valid on the routes from USA to Paris, with other routes slated for next year. \cite{AXAFizzy}
\end{enumerate}

\section{Problem Statement}
Creating an automated flight delay insurance platform that provides the buyer complete transparency about insurance rates and automates the manual process of claims reimbursement. The solution will consist of four sub-tasks, independent but connected to each other.
    \subsection{Flight data}
    A dataset consisting of every commercial flight that landed in Germany between December and May 2017, a six month interval. The dataset will also contain weather conditions of the airport at the time of flight landing.
    \subsection{Flight Delay prediction}
    A highly accurate prediction model for the flight delays. This model will be used in calculating the insurance payouts. If the model predicts a high probability of delay, the insurance payout will be lower. If the model predicts a low probability of delay, the insurance payout to the user will be higher.
    \subsection{Blockchain}
    The legal agreement for insurance will be created with the help of Blockchain. Blockchain will provide essential requirements for an insurance platform, Time-stamping and Non-Repudiation. Due to blockchain's immutable architecture, the integrity of customer data can also be ensured.
    \subsection{Website}
    A simple but secure web application that will enable a user to check out the insurance rates for the desired flight before purchasing, buy insurance and provide status of the insurance. There will be no claim process defined as no user input will be required for that. The claims process will be automatically triggered if the flight was delayed.

\section{Objective and Limitations}
The aim of the project, as already mentioned, is to create an insurance platform with modern technologies that is not possible with current business practices. But the service doesn't aim to provide any commercial services just yet. As insurance industry is heavily regulated, using technologies such as blockchain, which is not entirely regulated, would be ill-advised. This project is more of a proof of concept to show the strengths of blockchain as an insurance platform that will help and guide regulators in creating practical guidelines and regulations. The legal implications too have to be thoroughly studied and be taken into consideration before creating an insurance platform, not currently delved into detail by regulators and/or auditors.
\\Another limitation is that the prediction algorithm will consider only flights landing in Germany. The prediction algorithm might be a lot different if flights landing in other countries are considered.
\\The website created being a prototype, there will be no transfer of money. The transactions necessary for prototype functionality will be simulated. For example, by default it is assumed that the user pays \EUR{10} for buying the insurance and the payouts shown to the user for different delays will be based on that insurance amount. The claim process will also email the user about insurance amount they will be payed using Emails, but any transfer of money will not take place.